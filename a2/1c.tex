\documentclass{article}
\usepackage{mathtools}
\usepackage{amssymb}
\begin{document}

Given
\begin{align*}
  K(\mathbf{u},\mathbf{v}) &= (1 + \mathbf{u^Tv})^2\\
  &= 1+2\mathbf{u^Tv}+(\mathbf{u^Tv})^2\\
  &= 1+2\mathbf{u^Tv}+(\mathbf{u^Tv})^2\\
  &= 1+2u_1v_1+2u_2v_2+(u_1^2v_1^2+2u_1v_1u_2+v_2 + u_2^2v_2^2)
\end{align*}

Let us realize that this kernel suggests a feature space
\([1, \sqrt{2}u_1, \sqrt{2}u_2, u_1^2, u_2^2, \sqrt{2}u_1u_2]\). For simplicity,
we adapt this feature space more generally as \([1, x_1, x_2, x_1^2, x_2^2, x_1x_2]\)
and drop the constant multipliers as suggested. Then, given an ellipse is defined by
\begin{align*}
  c(x_1-a)^2 + d(x_2-b)^2  &= 1\\
  cx_1-2acx_1 + ca_2^2 + dx_2^2 - 2dbx_2 + db^2 - 1 &= 0\\
\end{align*}
we wish to recycle the proof from 1b. To do this, we form the following vector
\begin{equation*}
  \mathbf{w} = [ca^2+db^2-1, -2ac, -2db, c, d, 0]
\end{equation*}
Then, we define \(y_i = -1 \) if a point lies within the ellipse, \(y_i = 1 \) otherwise.
We simply adopt the inequalities from 1b and claim that
\begin{align*}
  \mathbf{w^{T}x}+\beta & >0\textnormal{, if }\mathbf{x}\textnormal{ escapes the ellipse region}\\
  \mathbf{w^{T}x}+\beta & <0\textnormal{, if }\mathbf{x}\textnormal{ occupies the ellipse region}\\
  \mathbf{w^{T}x}+\beta & =0\textnormal{, if }\mathbf{x}\textnormal{ demarcates the ellipse region}
\end{align*}
which satisfies the separability constraint \(y_{i}(\mathbf{w^{T}x}+\beta)>0,\forall i\)
\end{document}
