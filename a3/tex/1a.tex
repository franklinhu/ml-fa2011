\documentclass{article}
\usepackage{mathtools}
\usepackage{amssymb}
\begin{document}

1a. 
Let us consider
\begin{align*}
B(q) &= -qlog(q)-(1-q)log(1-q)\\
\frac{dB}{dq} &= log(1-q) - log(q)\\
\frac{d^2B}{d^2q} &= \frac{1}{(q-1)q}\\
\end{align*}

Then, let \(q = \frac{p}{p+n}\). We wish to find a maxima in order to demonstrate \(H(S) = B(\frac{p}{p+n}) \leq 1\). Then, 

\begin{align*}
B'(\frac{p}{p+n})
&= log(1-\frac{p}{p+n}) - log(\frac{p}{p+n})\\
&= log(\frac{n}{p+n}) - log(\frac{p}{p+n})\\
&= log(\frac{\frac{n}{p+n}}{\frac{p}{p+n}})\\
&= log(\frac{n}{p}) = 0
\end{align*}
This shows that there exists an optima where \(n = p\) and we can verify that this point is a maximum by
\begin{align*}
B''(\frac{p}{p+n}) &= \frac{1}{(\frac{p}{p+n}-1)\frac{p}{p+n}}\\
\textnormal{Since \(n = p\),}\\
&= \frac{1}{(0.5-1)0.5} < 0
\end{align*}
Therefore, there exists a maximum when \(n = p\). Note that in this scenario,

\begin{align*}
H(S) 
&= B(\frac{p}{p+p})=B(0.5)\\
&= -0.5\cdot log(0.5)-0.5\cdot log(0.5)\\
&= -log(0.5)\\
&= 1
\end{align*}

which shows that the equality is achieved under said constraint.

\end{document}
